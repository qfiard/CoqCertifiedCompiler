\documentclass[a4paper,twoside,12pt]{article}

\usepackage{amsmath}
\usepackage{amsfonts}
\usepackage{amssymb}
\usepackage{fancyhdr}
\usepackage{stmaryrd}
\usepackage{textcomp}
\usepackage[french]{babel}
\usepackage{variations}
\usepackage{graphicx}
\usepackage{array}
\usepackage{verbatim}
\usepackage{bbm}
\usepackage{ifthen}
\usepackage[paper=a4paper,hmargin=28mm,top=8mm,bottom=52.5mm,includehead=true,twoside=true,headsep=5mm,driver=xetex,twoside=true,bindingoffset=6mm]{geometry}

%textpos : permet de placer des éléments par leur coordonnées absolues sur la page
% utilisé pour les entêtes et pied de pages

\usepackage[absolute]{textpos}

\setlength{\TPHorizModule}{1mm}
\setlength{\TPVertModule}{1mm}

% appendix : permet d'écrire en couleur

\usepackage[usenames,dvipsnames,table]{xcolor}

%% pdfpages : utilisé pour insérer la page de garde avant le document

\usepackage{pdfpages}

% caption : centre les légendes des figures

\usepackage[center]{caption}

% wrapfig : permet de créer des figures entourées de texte

\usepackage{wrapfig}

%placeins : permet d'obliger Latex à insérer les figures en attente avant de continuer

\usepackage{placeins}

% subcaption : permet de créer des figures avec plusieurs sous-figures

\usepackage{subcaption}

% layouts : affichage d'une longueur

\usepackage{layouts}

% tabularx : mise en forme des tableaux
\usepackage{color}
\usepackage{longtable}
\usepackage{colortbl,arydshln}
\usepackage{tabularx}

% eurosym : symbole euro

\usepackage{eurosym}

\usepackage{tocvsec2}

% xelatex
\usepackage{xltxtra}

\usepackage{lastpage}

\usepackage{xparse}
\usepackage{hyperref}

\usepackage[nonumberlist]{glossaries}
\newcommand{\str}[1]{\renewcommand{\arraystretch}{#1}}

\makeatletter
\let\tobi@kept@section\section
\RenewDocumentCommand{\section}{s}
  {\IfBooleanTF{#1}{\tobi@kept@section*}{\tobi@section}}
\NewDocumentCommand{\tobi@section}{o m o}
  {\IfNoValueTF{#3}% No short title for the header
     {\IfNoValueTF{#1}% No title for the toc
        {\tobi@kept@section{#2}}
        {\tobi@kept@section[#1]{#2}}%
     }
     {\IfNoValueTF{#1}% No title for the toc
        {\tobi@kept@section[#2]{#2\sectionmark{#3}}\sectionmark{#3}}
        {\tobi@kept@section[#1]{#2\sectionmark{#3}}\sectionmark{#3}}%
    }%
  }
\makeatother


\newcommand{\refbiblio}[1]{\cite{#1}}
\newcommand{\norefbiblio}[1]{\nocite{#1}}

\newcommand{\mymark}[1]{\markboth{\MakeUppercase{#1}}{\MakeUppercase{#1}}}



\newenvironment{intro}
{
\section*{Introduction}
\addcontentsline{toc}{section}{Introduction}
\mymark{Introduction}
}
{

}

\newenvironment{remerciements}
{
\section*{Remerciements}
\addcontentsline{toc}{section}{Remerciements}
\mymark{Remerciements}
}
{

}

\newenvironment{conclusion}
{
\section*{Conclusion}
\addcontentsline{toc}{section}{Conclusion}
\mymark{Conclusion}
}
{

}

\newenvironment{resume}
{
\section*{Résumé}
\addcontentsline{toc}{section}{Résumé}
\mymark{Résumé}
}
{

}


\newenvironment{sous-partie}[1]
{
\subsection{#1}
}
{

}

\newenvironment{sous-sous-partie}[1]
{
\subsubsection{#1}
}
{

}

\newenvironment{liste}
{
\vspace{0.2cm}
\begin{list}{$\bullet$\hspace{0.3cm}}{\leftmargin=1.4cm}
}
{
\end{list}
\vspace{0.2cm}
}

\newenvironment{oliste}
{
\vspace{0.2cm}
\begin{enumerate}
}
{
\end{enumerate}
\vspace{0.2cm}
}



\newenvironment{partie}[1]
{
\section{#1}
}
{

}

\newcounter{numannexe}
\newenvironment{annexes}
{
\newpage
\thispagestyle{empty}
\mbox{}
\newpage
\addcontentsline{toc}{section}{Annexes}
\mymark{Annexes}
\vspace*{\fill}
\begin{center}
\Huge{\textbf{Annexes}}
\end{center}
\vspace*{\fill}
\newpage
\setcounter{numannexe}{0}

\renewenvironment{partie}[1]
{
	\refstepcounter{subsection}
	\subsection*{\arabic{subsection}. ##1}
}
{
}

}
{

}

\newenvironment{annexe}[1]
{
\setcounter{subsection}{0}
\refstepcounter{numannexe}
\addcontentsline{toc}{subsection}{Annexe \arabic{numannexe} : #1}
\mymark{Annexe \arabic{numannexe} : #1}
\section*{Annexe \arabic{numannexe} : #1}
}
{
	\FloatBarrier
}

\newenvironment{sous-annexe}[1]
{
\begin{subappendices}
\subsection{#1}
}
{
\end{subappendices}
}

% Permet de faire des cadres dans le document
\newcommand{\cadre}[1]{\renewcommand{\arraystretch}{0.4}\begin{array}{!{\vline}c!{\vline}}\hline\\#1\\\\\hline\end{array}}
\newcommand{\cadret}[1]{\renewcommand{\arraystretch}{0.4}\begin{tabular}{!{\vline}c!{\vline}}\hline\\#1\\\\\hline\end{tabular}}
\newcommand{\semicadre}[1]{\renewcommand{\arraystretch}{0.3}\begin{array}{c!{\vline}}#1\\\\\hline\end{array}}

% Commandes mathématiques

\newcommand{\equivalent}[2]{\!\!\renewcommand{\arraystretch}{1}\begin{array}[t]{c}\Huge\sim\\^{#1\rightarrow#2}\end{array}\!\!}
\newcommand{\tend}[2]{\!\renewcommand{\arraystretch}{1}\begin{array}[t]{c}-\!-\!\!\!\longrightarrow\\^{#1\rightarrow#2} \\ [-1.5ex]\end{array}\!}
\newcommand{\egal}[2]{\!\!\renewcommand{\arraystretch}{1}\begin{array}[t]{c}=\\^{#1\rightarrow#2}\end{array}\!\!}
\newcommand{\fin}{\vspace{0.2cm}\\}
\newcommand{\finq}{\vspace{0.5cm}\\}
\newcommand{\sh}{\mathrm{sh}\,}
\newcommand{\abs}[1]{\left\vert#1\right\vert }
\newcommand{\ch}{\mathrm{ch}\,}

\renewcommand{\o}[1]{\mathrm{o}\!\left(#1\right)}
\newcommand{\etoile}{\hspace*{1cm}$\star$\hspace*{0.5cm}}
\renewcommand{\th}{\mathrm{th}\,}
\renewcommand{\arcsin}{\mathrm{Arcsin}\,}
\renewcommand{\arccos}{\mathrm{Arccos}\,}
\newcommand{\argsh}{\mathrm{Argsh}\,}
\newcommand{\argch}{\mathrm{Argch}\,}
\newcommand{\argth}{\mathrm{Argth}\,}
\newcommand{\rg}{\mathrm{rg}\,}
\renewcommand{\arctan}{\mathrm{Arctan}\,}
\newcommand{\sq}{\hspace*{1.4cm}\stepcounter{sq}(\alph{sq})\hspace*{0.5cm}}
\newcommand{\rcl}{\begin{array}{rcl}}
\newcommand{\ea}{\end{array}}
\renewcommand{\tfrac}[2]{\textstyle\frac{#1}{#2}}
\newcommand{\Cl}[1]{$C^{#1}$}
\newcommand{\mathCl}[1]{C^{#1}}
\renewcommand{\t}[1]{\tilde{#1}}
\renewcommand{\l}{\lambda}
\newcommand{\ds}{\displaystyle}
\newcommand{\R}{\mathbb{R}}
\newcommand{\Q}{\mathbb{Q}}
\newcommand{\Z}{\mathbb{Z}}
\newcommand{\N}{\mathbb{N}}
\newcommand{\Ker}{\mathrm{Ker}\,}
\newcommand{\Vect}{\mathrm{Vect}}
\renewcommand{\lvert}{\left\vert}
\renewcommand{\Im}{\mathrm{Im}\,}
\renewcommand{\rvert}{\right\vert}
\newcommand{\mnk}{\mathcal{M}_n(K)}
\newcommand{\mnc}{\mathcal{M}_n(C)}
\newcommand{\ppcm}{\mathrm{ppcm}}
\newcommand{\Tr}{\mathrm{Tr}\,}
\renewcommand{\t}[1]{^t\!#1}
\newcommand{\scal}[2]{\left\langle #1|#2\right\rangle}
\newcommand{\scalindice}[4]{\phantom{\langle}_{#3}\!\left\langle #1|#2\right\rangle_{#4}}
\newcommand{\p}[1]{\left( #1 \right)}
\newcommand{\crochet}[1]{\left[ #1 \right]}
\newcommand{\nr}[1]{\left\|\,#1\,\right\|}
\newcommand{\tab}{\hspace*{1cm}}

\newcommand{\esp}[1]{\mathbb{E}\!\crochet{#1}}
\newcommand{\espcond}[2]{\mathbb{E}_{#1}\!\crochet{#2}}

\renewcommand{\P}[1]{\mathbb{P}\!\p{#1}}
\newcommand{\Pcond}[2]{\mathbb{P}_{#1}\!\p{#2}}

\renewcommand{\binom}[2]{\left(\begin{array}{c}#1\\#2\end{array}\right)}

\newcommand{\car}[1]{\mathbf{1}_{#1}}

\newcommand{\matdd}[4]{\left({\begin{array}{cc} #1 & #2\\ #3 & #4 \ea}\right)\vspace{0.05cm}}

\newcommand{\supp}[1]{\text{supp}\left(#1\right)}

\renewcommand{\ge}{\geqslant}
\renewcommand{\le}{\leqslant}

\newcommand{\lebesgue}{\mathcal{L}}

\newcommand{\Drond}{\mathcal{D}}

\renewcommand{\Re}{\mathcal{R}e}
\newcommand{\obs}[1]{\hat{#1}}
\newcommand{\ket}[1]{\vert #1 \rangle}
\newcommand{\bra}[1]{\langle #1 \vert}
\newcommand{\braindice}[2]{\!\phantom{\langle}_{#2}\!\left\langle #1\right\vert}

\newcommand{\prodscal}[2]{\left\langle#1,#2\right\rangle}

\newcommand{\vect}[2]{\left(\str{1}\begin{array}{cc}#1 & #2 \ea\right)}
\newcommand{\vecttrans}[2]{\left(\str{1}\begin{array}{c}#1 \\ #2 \ea\right)}

\renewcommand{\v}[1]{\underline{#1}}

\newcommand{\Dp}[2]{\dfrac{\partial #1}{\partial #2}}
\newcommand{\grad}{\mathrm{grad}\,}
\newcommand{\vgrad}{\v{\mathrm{grad}}\,}


\usepackage{listings}

\renewcommand{\thesection}{\Roman{section}}

\pagestyle{fancy}

\newlength{\headrulemargintop}
\setlength{\headrulemargintop}{10.15mm}


\newlength{\headbottommargin}

\setlength{\headheight}{2cm}
\setlength{\footskip}{2.65cm}

\renewcommand{\headrulewidth}{0pt}
\renewcommand{\footrulewidth}{0pt}

\setlength{\headsep}{6mm}
\addtolength{\headrulemargintop}{-\headsep}
\setlength{\headbottommargin}{\headsep}
\addtolength{\headheight}{\headsep}
\addtolength{\headsep}{-4mm}
\addtolength{\headbottommargin}{-\headsep}

\renewcommand{\headrule}{\vspace*{-\headrulemargintop}}
\renewcommand{\headrule}{\vspace*{-\headrulemargintop}}

\renewcommand{\sectionmark}[1]{\markright{\thesection\ \MakeUppercase{#1}}}
\renewcommand{\subsectionmark}[1]{\markright{\thesubsection\ \MakeUppercase{#1}}}

\fancyhead{}
\fancyfoot{}

\definecolor{pantone}{rgb}{0.,0.26274509803922,0.37647058823529}

\newlength{\logoskyline}
\setlength{\logoskyline}{3cm}
\newlength{\logoskylineleft}
\setlength{\logoskylineleft}{16.965250619335cm}
%\setlength{\logoskylineleft}{21cm}
%\addtolength{\logoskylineleft}{-\logoskyline}
%\addtolength{\logoskylineleft}{-6mm}

\fancyhead[LO]{%
\color{pantone}
\small
\str{0.9}
\begin{tabular}{@{}l@{}}
{\bf %\fontspec{Gill Sans Light}
INF551 - Raisonnement assisté par ordinateur}\\
{\bf
%\fontspec{Gill Sans Light}
Rapport de projet - Compilateur vérifié en Coq}
\end{tabular}\\\vspace*{3.1mm}
\str{0.9}\begin{tabular}{@{}l@{}}
{\selectfont
%\fontspec{Gill Sans Light}
Dorian Nogneng}\\{\selectfont %\fontspec{Gill Sans Light}
Quentin Fiard}
\end{tabular}
\vspace*{\headbottommargin}
\begin{textblock}{5}(12,8)
\includegraphics[height=2.05cm]{./logo_x.eps}
\end{textblock}
\begin{textblock}{200}(34,19)
\hrule width 130.2486772486772mm
\end{textblock}}

\fancyhead[LE]{%
\color{pantone}
\small
\str{0.9}
\begin{tabular}{@{}l@{}}
{\bf %\fontspec{Gill Sans Light}
INF551 - Raisonnement assisté par ordinateur}\\
{\bf
%\fontspec{Gill Sans Light}
Rapport de projet - Compilateur vérifié en Coq}
\end{tabular}\\\vspace*{3.1mm}
\str{0.9}\begin{tabular}{@{}l@{}}
{\selectfont
%\fontspec{Gill Sans Light}
Dorian Nogneng}\\{\selectfont %\fontspec{Gill Sans Light}
Quentin Fiard}
\end{tabular}
\vspace*{\headbottommargin}
\begin{textblock}{5}(6,8)
\includegraphics[height=2.05cm]{./logo_x.eps}
\end{textblock}
\begin{textblock}{200}(28,19)
\hrule width 130.2486772486772mm
\end{textblock}}

\fancyfoot[L]{\fancyplain{}\rightmark}

\fancyfoot[R]{\thepage\ / \pageref{LastPage}}

\newcommand{\definition}[2]{{\bf #1} : #2}

\makeglossaries

\renewcommand{\glsnamefont}[1]{\makefirstuc{#1}}

%\newglossaryentry{wattpic}
%{
%  name=watt-pic,
%  description={Puissance nominale produite par un système photovoltaïque sous %l'illumination maximale du site considéré}
%}

\begin{document}
%\includepdf{img/page0.pdf}

\setcounter{page}{1}

\clearpage

\tableofcontents

\clearpage

\begin{intro}

\begin{paragraph}{Description du projet\vspace{0.2cm}\\}
Ce projet consiste en la mise au point et la certification en Coq d'un compilateur pour un mini-langage fonctionnel. Nous allons dans ce rapport présenter le langage considéré, la machine abstraite de destination et le fonctionnement général du compilateur développé.
\end{paragraph}

\end{intro}

\begin{partie}{Langage étudié}

\begin{sous-partie}{Indices de De Bruijn et manipulation de $\lambda$-termes}

\begin{paragraph}{}
Le langage étudié est un modèle très simple des langages fonctionnels. Les termes de ce langage sont définis par la grammaire suivante :
\end{paragraph}
$$\begin{array}{rcll}
t&::=&x&\mbox{variable}\\
&|&\lambda\,x\cdot t&\mbox{fonction associant t à x}\\
&|&t\;u&\mbox{application de la fonction t à u}
\ea$$
\begin{paragraph}{}
Pour ôter l'indétermination liée à l'$\alpha$-équivalence de termes, l'énoncé propose d'utiliser les indices de De Bruijn : chaque variable liée est remplacée par un entier représentant le nombre de lieurs $\lambda$ la séparant de l'abstraction à laquelle elle est liée. Un exemple de terme de ce langage avec indices de Bruijn est par exemple $\lambda\cdot\lambda\cdot(0\;1)$, où le 1 est associé au $\lambda$ le plus externe.
\end{paragraph}

\end{sous-partie}

\begin{sous-partie}{Formalisation en Coq}

\begin{paragraph}{}
Les termes avec indices de De Bruijn sont représentés par le type \verb|DBT| en Coq. Ce type est défini par l'inductif suivant :
\end{paragraph}\vspace{-0.8cm}
\begin{center}
\begin{tabular}{p{5cm}}
\begin{verbatim}
Inductive DBT:Type :=
| Var : nat -> DBT
| Fun : DBT -> DBT
| Appl : DBT -> DBT -> DBT.
\end{verbatim}
\end{tabular}
\end{center}\vspace{-1cm}
\begin{paragraph}{}
Un terme de type \verb|Var x| représente une variable d'indice de De Bruijn $x\in\N$, \verb|Fun t| est la fonction qui à un terme $u$ associe le terme $t[0\leftarrow u]$, et \verb|Appl t1 t2| représente le terme $t1\;t2$, application de $t1$ à $t2$.
\end{paragraph}

\end{sous-partie}

\begin{sous-partie}{Formalisation des termes clos sous forme d'un prédicat}

\begin{paragraph}{}
Comme suggéré dans l'énoncé, nous avons procédé en deux étapes pour définir le prédicat ``Terme clos" à partir de la définition précédente des termes avec indices de De Bruijn.
\end{paragraph}
\begin{paragraph}{}
Nous avons commencé par formaliser le prédicat $C[n](t)$ suivant en Coq : pour tout $t$ terme avec indices de De Bruijn, $t$ satisfait $C[n](t)$ si et seulement si toutes les variables libres de $t$ sont d'indices $<n$. Ce prédicat est dénommé \verb|n_free t n| dans le code Coq (\verb|n_free t n = true| si et seulement si $C[n](t)$ est satisfait).
\end{paragraph}
\begin{paragraph}{}
Pour calculer ce prédicat, on commence par définir un prédicat auxiliaire\\\verb|n_k_free : DBT -> nat -> nat|, tel que \verb|n_k_free t n k = true| si et seulement si
toutes les variables libres de $t$ sont $<n$, en faisant l'hypothèse additionnelle que toutes les variables $<k$ sont liées. On a alors pour tout $t$ et $n$,
$$\verb|n_free t n = n_k_free t n 0|$$
et on peut facilement calculer \verb|n_k_free| inductivement sur $t$, en utilisant les relations suivantes :
\end{paragraph}
$$\begin{array}{rcl}
\verb€n_k_free (Var x) n k €&=&\verb€ (x<n)||(x<k)€\\
\verb€n_k_free (Fun t) n k €&=&\verb€ n_k_free t n (k+1)€\\
\verb€n_k_free (Appl t1 t2) n k €&=&\verb€ (n_k_free t1 n k) && (n_k_free t2 n k)€
\ea$$
\begin{paragraph}{}
On a ensuite montré que pour tout \verb|n|, \verb|n_free t n -> n_free t (n+1)| en démontrant tout d'abord cette propriété sur \verb|n_k_free|. La définition du prédicat ``Terme clos" peut alors simplement s'écrire \verb| closed t := n_free t 0|.
\end{paragraph}
\end{sous-partie}
\begin{sous-partie}{Substitution dans un terme}
\begin{sous-sous-partie}{Substitution unique}
\begin{paragraph}{}
On se donne un terme \verb|t|, un entier \verb|n| et un autre terme \verb|u|. On cherche à substituer \verb|u| à la variable \verb|n| dans \verb|t|, ce pour chaque occurrence libre de \verb|n|.
\end{paragraph}
\begin{paragraph}{}
Là encore, cette fonction est définie par induction sur \verb|t|. Les cas 1 et 3 sont simples à traiter :
\end{paragraph}
$$\begin{array}{rcl}
\verb€substitute_one (Var x) n u €&=&\begin{array}[t]{|cl}
\verb€ u€&\mbox{si }\verb|n=x|\\
\verb€ Var x€&\mbox{sinon}
\ea\\
\verb€substitute_one (Appl t1 t2) n u €&=&\verb|Appl (substitute_one t1 n u)| \\
&&\hspace{1.5cm}\verb|(substitute_one t2 n u)|\\
\ea$$
\begin{paragraph}{}
Le cas où \verb|t=Fun t1| est plus complexe à gérer. En effet, si 0 est une variable libre de \verb|u|, alors si on substitue directement \verb|u| à \verb|n| dans  \verb|t1|, on capture la variable libre 0 de \verb|u| par ce $\lambda$ additionnel. Pour préserver la liberté des variables libres de \verb|u|, une méthode consiste à augmenter de 1 la valeur de toutes les variables libres de \verb|u| avant la substitution. C'est la méthode que nous avons choisi d'utiliser.
\end{paragraph}
\begin{paragraph}{}
On appelle \verb|shift| cette opération. Pour la définir, nous avons utilisé une fonction \verb|add_n_to_free| qui ajoute une constante \verb|n| à toutes les variables libres d'un terme, que l'on a définit par induction sur la taille des terme, de façon similaire à la définition de \verb|n_free| (on utilise une fonction auxiliaire où on fait une hypothèse supplémentaire sur les variables liées). On a alors :
\end{paragraph}
$$\verb€substitute_one (Fun t1) n u = Fun (substitute_one t1 n (shift u))€$$
\begin{paragraph}{}
Nous avons alors pu démontrer formellement grâce à Coq que toute substitution dans un terme fermé $t$ renvoie ce même terme.
\end{paragraph}
\end{sous-sous-partie}
\begin{sous-sous-partie}{Substitution multiple}
\begin{paragraph}{}
On cherche maintenant à formaliser la substitution $t[i\leftarrow u_0\dots u_n]$, où $t, u_0, \dots, u_n$ sont des termes et $i, \dots, i+n-1$ des variables libres de $t$. Cette substitution est définie par récurrence en utilisant :
$$t[i\leftarrow u_0\dots u_n] = (t[i\leftarrow u_0])[i+1\leftarrow u_1,\dots,u_n]$$
Nous avons alors prouvé en utilisant Coq que :
$$\begin{array}{rcl}
t[] &=& t\\
t[i\leftarrow u] &=& t\quad\mbox{si $t$ satisfait $C[i](t)$}
\ea$$
Nous n'avons à l'heure où nous écrivons ce rapport pas encore démontré que
$$t[i\leftarrow u_0\dots u_n] = t[i\leftarrow u_0\dots u_n][i\leftarrow u_0] \quad\mbox{si $\forall k\le i, C[i](u_k)$}$$
mais espérons le faire avant la présentation orale du mardi 18 décembre 2012.
\end{paragraph}
\end{sous-sous-partie}
\end{sous-partie}

\end{partie}

\begin{partie}{Machine abstraite, modèle d'exécution}

\begin{sous-partie}{$\beta$-réduction des $\lambda$-termes}

\begin{paragraph}{}
Pour formaliser la notion de $\beta$-réductibilité d'un $\lambda$-terme en un autre, on commence par définir la réductibilité en une étape : la propriété \verb|reduces_in_one_step a b| est vraie ssi a se réduit en b en une étape de réduction. On utilise alors les propriétés suivantes :
\end{paragraph}
$$\begin{array}{l}
\verb|reduces_in_one_step (|\lambda\verb| t u) (t[0<-u])|\\
\verb|reduces_in_one_step t u =>|\\
\hspace{1cm}\bullet\hspace{0.5cm}\verb|reduces_in_one_step (t v) (u v)|\\
\hspace{1cm}\bullet\hspace{0.5cm}\verb|reduces_in_one_step (v t) (v u)|\\
\hspace{1cm}\bullet\hspace{0.5cm}\verb|reduces_in_one_step (|\lambda\verb| t) (|\lambda\verb| u)|
\ea$$
\begin{paragraph}{}
On définit ensuite par induction la propriété générale de $\beta$-réductibilité :\\
\verb|reduces_in_few_steps a b| est vraie ssi a se réduit en b en un nombre fini de réductions. On a alors les propriétés suivantes :
\end{paragraph}
$$\begin{array}{rl}
\mbox{Réflexivité :}&\verb|reduces_in_few_steps t t|\\
\mbox{Transitivité :}&\verb|reduces_in_few_steps t u && reduces_in_one_step u v =>|\\
&\hspace{2cm}\verb|reduces_in_few_steps t v|\\
\ea$$
\begin{paragraph}{}
Après avoir démontré ces propriétés en Coq, nous avons montré que les propriété de clôture par contexte se propagent à \verb|reduces_in_few_steps| par induction sur la propriété \verb|reduces_in_few_steps|.
\end{paragraph}
\end{sous-partie}
\begin{sous-partie}{Machine abstraite de Krivine}
\begin{paragraph}{}
La machine abstraite de Krivine est un langage possédant 3 instructions :
\end{paragraph}
\begin{liste}
\item\verb|Access n| : aller à l'élément numéro n de l'environnement
\item\verb|Grab| : prendre le sommet de la pile dans l'environnement
\item\verb|Push c| : rajouter le bloc d'instructions c dans la pile
\end{liste}
\begin{paragraph}{}
L'état d'une telle machine est alors caractérisé par la donnée d'un bloc d'instructions à exécuter, un environnement et une pile.
\end{paragraph}
\begin{paragraph}{}
Nous avons tout d'abord défini ces états en Caml :
$$
\begin{array}{l}
\verb€type instruction = Access of int| Grab |Push of code€\\
\verb€and code == instruction list;;(*instruction list*)€\\
\verb€type environment = Empty | Node of (code*environment) list;;€\\
\verb|type stack == environment;;|\\
\verb€type state = {mutable c:code;mutable e:environment;mutable s:stack};;€
\ea
$$
avant d'écrire une fonction récursive \verb|execute|, simulant l'exécution de la machine jusqu'à la fin des instructions.
\end{paragraph}
\begin{paragraph}{}
En Coq, nous avons écrit une fonction \verb|execute_one| qui exécute une étape de la machine de Krivine, étant donné l'instruction actuelle (la terminaison de \verb|execute| n'est pas immédiate, et il est donc difficile de la définir récursivement en Coq, en utilisant des sous-parties strictes de l'état de la machine).
\end{paragraph}
\begin{paragraph}{}
Nous avons ensuite écrit la fonction de compilation, qui traduit un terme en écriture de De Bruijn en un code de machine de Krivine, par induction sur les termes.
\end{paragraph}
\end{sous-partie}
\end{partie}

\begin{conclusion}
\begin{paragraph}{}
En conclusion, ce projet nous a permis d'implémenter un compilateur traduisant un modèle simple de $\lambda$-calcul en langage machine de Krivine. Nous n'avons au moment de l'écriture de ce rapport pas encore eu le temps d'étudier la certification du compilateur en tant que telle, mais espérons pouvoir le faire avant la date de la soutenance, afin de pouvoir mettre en valeur l'intérêt de Coq dans un tel cadre.
\end{paragraph}
\end{conclusion}

%\glsaddall
%\printglossary[title={Glossaire\markright{GLOSSAIRE}}]
%\addcontentsline{toc}{section}{Glossaire}

\end{document}