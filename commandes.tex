\newcommand{\str}[1]{\renewcommand{\arraystretch}{#1}}

\makeatletter
\let\tobi@kept@section\section
\RenewDocumentCommand{\section}{s}
  {\IfBooleanTF{#1}{\tobi@kept@section*}{\tobi@section}}
\NewDocumentCommand{\tobi@section}{o m o}
  {\IfNoValueTF{#3}% No short title for the header
     {\IfNoValueTF{#1}% No title for the toc
        {\tobi@kept@section{#2}}
        {\tobi@kept@section[#1]{#2}}%
     }
     {\IfNoValueTF{#1}% No title for the toc
        {\tobi@kept@section[#2]{#2\sectionmark{#3}}\sectionmark{#3}}
        {\tobi@kept@section[#1]{#2\sectionmark{#3}}\sectionmark{#3}}%
    }%
  }
\makeatother


\newcommand{\refbiblio}[1]{\cite{#1}}
\newcommand{\norefbiblio}[1]{\nocite{#1}}

\newcommand{\mymark}[1]{\markboth{\MakeUppercase{#1}}{\MakeUppercase{#1}}}



\newenvironment{intro}
{
\section*{Introduction}
\addcontentsline{toc}{section}{Introduction}
\mymark{Introduction}
}
{

}

\newenvironment{remerciements}
{
\section*{Remerciements}
\addcontentsline{toc}{section}{Remerciements}
\mymark{Remerciements}
}
{

}

\newenvironment{conclusion}
{
\section*{Conclusion}
\addcontentsline{toc}{section}{Conclusion}
\mymark{Conclusion}
}
{

}

\newenvironment{resume}
{
\section*{Résumé}
\addcontentsline{toc}{section}{Résumé}
\mymark{Résumé}
}
{

}


\newenvironment{sous-partie}[1]
{
\subsection{#1}
}
{

}

\newenvironment{sous-sous-partie}[1]
{
\subsubsection{#1}
}
{

}

\newenvironment{liste}
{
\vspace{0.2cm}
\begin{list}{$\bullet$\hspace{0.3cm}}{\leftmargin=1.4cm}
}
{
\end{list}
\vspace{0.2cm}
}

\newenvironment{oliste}
{
\vspace{0.2cm}
\begin{enumerate}
}
{
\end{enumerate}
\vspace{0.2cm}
}



\newenvironment{partie}[1]
{
\section{#1}
}
{

}

\newcounter{numannexe}
\newenvironment{annexes}
{
\newpage
\thispagestyle{empty}
\mbox{}
\newpage
\addcontentsline{toc}{section}{Annexes}
\mymark{Annexes}
\vspace*{\fill}
\begin{center}
\Huge{\textbf{Annexes}}
\end{center}
\vspace*{\fill}
\newpage
\setcounter{numannexe}{0}

\renewenvironment{partie}[1]
{
	\refstepcounter{subsection}
	\subsection*{\arabic{subsection}. ##1}
}
{
}

}
{

}

\newenvironment{annexe}[1]
{
\setcounter{subsection}{0}
\refstepcounter{numannexe}
\addcontentsline{toc}{subsection}{Annexe \arabic{numannexe} : #1}
\mymark{Annexe \arabic{numannexe} : #1}
\section*{Annexe \arabic{numannexe} : #1}
}
{
	\FloatBarrier
}

\newenvironment{sous-annexe}[1]
{
\begin{subappendices}
\subsection{#1}
}
{
\end{subappendices}
}

% Permet de faire des cadres dans le document
\newcommand{\cadre}[1]{\renewcommand{\arraystretch}{0.4}\begin{array}{!{\vline}c!{\vline}}\hline\\#1\\\\\hline\end{array}}
\newcommand{\cadret}[1]{\renewcommand{\arraystretch}{0.4}\begin{tabular}{!{\vline}c!{\vline}}\hline\\#1\\\\\hline\end{tabular}}
\newcommand{\semicadre}[1]{\renewcommand{\arraystretch}{0.3}\begin{array}{c!{\vline}}#1\\\\\hline\end{array}}

% Commandes mathématiques

\newcommand{\equivalent}[2]{\!\!\renewcommand{\arraystretch}{1}\begin{array}[t]{c}\Huge\sim\\^{#1\rightarrow#2}\end{array}\!\!}
\newcommand{\tend}[2]{\!\renewcommand{\arraystretch}{1}\begin{array}[t]{c}-\!-\!\!\!\longrightarrow\\^{#1\rightarrow#2} \\ [-1.5ex]\end{array}\!}
\newcommand{\egal}[2]{\!\!\renewcommand{\arraystretch}{1}\begin{array}[t]{c}=\\^{#1\rightarrow#2}\end{array}\!\!}
\newcommand{\fin}{\vspace{0.2cm}\\}
\newcommand{\finq}{\vspace{0.5cm}\\}
\newcommand{\sh}{\mathrm{sh}\,}
\newcommand{\abs}[1]{\left\vert#1\right\vert }
\newcommand{\ch}{\mathrm{ch}\,}

\renewcommand{\o}[1]{\mathrm{o}\!\left(#1\right)}
\newcommand{\etoile}{\hspace*{1cm}$\star$\hspace*{0.5cm}}
\renewcommand{\th}{\mathrm{th}\,}
\renewcommand{\arcsin}{\mathrm{Arcsin}\,}
\renewcommand{\arccos}{\mathrm{Arccos}\,}
\newcommand{\argsh}{\mathrm{Argsh}\,}
\newcommand{\argch}{\mathrm{Argch}\,}
\newcommand{\argth}{\mathrm{Argth}\,}
\newcommand{\rg}{\mathrm{rg}\,}
\renewcommand{\arctan}{\mathrm{Arctan}\,}
\newcommand{\sq}{\hspace*{1.4cm}\stepcounter{sq}(\alph{sq})\hspace*{0.5cm}}
\newcommand{\rcl}{\begin{array}{rcl}}
\newcommand{\ea}{\end{array}}
\renewcommand{\tfrac}[2]{\textstyle\frac{#1}{#2}}
\newcommand{\Cl}[1]{$C^{#1}$}
\newcommand{\mathCl}[1]{C^{#1}}
\renewcommand{\t}[1]{\tilde{#1}}
\renewcommand{\l}{\lambda}
\newcommand{\ds}{\displaystyle}
\newcommand{\R}{\mathbb{R}}
\newcommand{\Q}{\mathbb{Q}}
\newcommand{\Z}{\mathbb{Z}}
\newcommand{\N}{\mathbb{N}}
\newcommand{\Ker}{\mathrm{Ker}\,}
\newcommand{\Vect}{\mathrm{Vect}}
\renewcommand{\lvert}{\left\vert}
\renewcommand{\Im}{\mathrm{Im}\,}
\renewcommand{\rvert}{\right\vert}
\newcommand{\mnk}{\mathcal{M}_n(K)}
\newcommand{\mnc}{\mathcal{M}_n(C)}
\newcommand{\ppcm}{\mathrm{ppcm}}
\newcommand{\Tr}{\mathrm{Tr}\,}
\renewcommand{\t}[1]{^t\!#1}
\newcommand{\scal}[2]{\left\langle #1|#2\right\rangle}
\newcommand{\scalindice}[4]{\phantom{\langle}_{#3}\!\left\langle #1|#2\right\rangle_{#4}}
\newcommand{\p}[1]{\left( #1 \right)}
\newcommand{\crochet}[1]{\left[ #1 \right]}
\newcommand{\nr}[1]{\left\|\,#1\,\right\|}
\newcommand{\tab}{\hspace*{1cm}}

\newcommand{\esp}[1]{\mathbb{E}\!\crochet{#1}}
\newcommand{\espcond}[2]{\mathbb{E}_{#1}\!\crochet{#2}}

\renewcommand{\P}[1]{\mathbb{P}\!\p{#1}}
\newcommand{\Pcond}[2]{\mathbb{P}_{#1}\!\p{#2}}

\renewcommand{\binom}[2]{\left(\begin{array}{c}#1\\#2\end{array}\right)}

\newcommand{\car}[1]{\mathbf{1}_{#1}}

\newcommand{\matdd}[4]{\left({\begin{array}{cc} #1 & #2\\ #3 & #4 \ea}\right)\vspace{0.05cm}}

\newcommand{\supp}[1]{\text{supp}\left(#1\right)}

\renewcommand{\ge}{\geqslant}
\renewcommand{\le}{\leqslant}

\newcommand{\lebesgue}{\mathcal{L}}

\newcommand{\Drond}{\mathcal{D}}

\renewcommand{\Re}{\mathcal{R}e}
\newcommand{\obs}[1]{\hat{#1}}
\newcommand{\ket}[1]{\vert #1 \rangle}
\newcommand{\bra}[1]{\langle #1 \vert}
\newcommand{\braindice}[2]{\!\phantom{\langle}_{#2}\!\left\langle #1\right\vert}

\newcommand{\prodscal}[2]{\left\langle#1,#2\right\rangle}

\newcommand{\vect}[2]{\left(\str{1}\begin{array}{cc}#1 & #2 \ea\right)}
\newcommand{\vecttrans}[2]{\left(\str{1}\begin{array}{c}#1 \\ #2 \ea\right)}

\renewcommand{\v}[1]{\underline{#1}}

\newcommand{\Dp}[2]{\dfrac{\partial #1}{\partial #2}}
\newcommand{\grad}{\mathrm{grad}\,}
\newcommand{\vgrad}{\v{\mathrm{grad}}\,}
